\chapter{Introduction} \label{chap:intro}

\section{Background}

Each year, landslides cause more than 5000 deaths \cite{perkins2012death}. The frequency and severity of flooding continue to escalate due to the effects of drought and global warming, making landslides an ongoing concern. This work aims to provide a tool to understand better and make predictions of the landslide process, including granular media, water, and air. The final goal is to create simulations of the process efficient and reproducible. 

This work proposes a new method for simulation of the process of landslides falling into the water to understand the involved physical mechanisms better and get more accurate predictions of the resulting impact. The proposed method will use computational fluid dynamics techniques to simulate the behavior of small particles, rocks, and boulders of arbitrary shape as they interact with each other and with the water. By improving our understanding of this process, \textbf{we} hope to provide valuable insights that can inform risk management strategies and help reduce the impact of landslides on communities and infrastructure.

The occurrence of landslides in coastal areas is a complex phenomenon. For accurate simulation, it is expected in the literature to use a fluid-structure interaction (FSI) \cite{belytschko1980fsi} approach, wherein the fluid component is modeled using computational fluid dynamics (CFD) methods and the granular media component is modeled using solid mechanics methods. The challenge lies in combining these two components in a way that facilitates communication between them. The interdependence between the fluid and granular media makes the modeling process time-consuming and complex. This research aims to develop an approach for effectively integrating these two components to improve the accuracy and efficiency of landslide simulations. Specifically, the proposed method will utilize advanced numerical techniques to accurately capture the fluid-granular media interaction and ensure that both components communicate seamlessly with each other. The outcome of this research could be applied to hazard assessment and risk management in areas prone to landslides. Although a couple of works were published in recent years \cite{mao2020resolved} \cite{shen2022resolved}, the current position not only uses a different approach for computational simulations but also uses open-source code and libraries with a goal being reproducible \cite{NAS2019}. Below, I will provide the three aims of this work and the methods that will be used to achieve them:

\textbf{Aim 1: Choose an efficient method to run simulations.}

In order to select the most appropriate and efficient method for running simulations, several key aspects were considered:
\begin{enumerate}
    \item {\textbf Coupling method:} Identify a suitable method for coupling the solid and fluid simulations, considering the characteristics of a landslide and granular media movement. This involved comparing different coupling techniques, such as one-way and two-way coupling, as well as investigating different levels of coupling, such as resolved and unresolved CFD-DEM approaches.
    \item {\textbf Free surface reconstruction:} It is necessary to evaluate various techniques for accurately capturing the free surface in multiphase flow simulations. This includes comparing different methods for free-surface reconstruction in terms of their accuracy, computational efficiency, and ease of implementation in the final solver.
    \item {\textbf Stability and robustness:} Assess the stability and robustness of the chosen methods, especially when considering the influence of the coupling between solid and fluid simulations. Stability and robustness may involve analyzing the convergence properties and sensitivity to numerical parameters such as time step size and grid resolution.
    \item {\textbf Scalability and parallelism:} Investigate the scalability and parallel performance of the chosen methods, ensuring that they can effectively utilize available computational resources and accommodate large-scale simulations.
    \item {\textbf Implementation complexity:} Consider the complexity of implementing the chosen methods within an existing simulation framework or developing a new one. The challenges of implementing include assessing the compatibility of the methods with existing software libraries and tools, as well as the effort required for code development and maintenance.
\end{enumerate}

By carefully evaluating these factors, the most efficient method for running simulations can be chosen, enabling accurate and reliable results while minimizing computational costs.

\textbf{Aim 2: Implement and run simulations using chosen method.} 

The implementation could be done by incorporating third-party libraries or modifying existing ones. After implementation, the model needs to be supported by a verification and validation procedure, which includes a parallel option for simulation to optimize performance. Once the implementation is complete, the results need to be verified and analyzed. The stability of the solver is supported by the grid convergence study as well. Last step to run application for a real-world problem to test its effectiveness and usefulness. Similarly to experiments where a granular mass fell into the water as in works \cite{mao2020resolved} or \cite{shen2022resolved}.

\textbf{Aim 3: Make code developed solver efficient and reproducible.}

The final aim was to create repro-pack following a tradition of reproducible research at Barba's group \cite{barba2018terminologies}, following to checklist from the thesis of Mesnard \cite{Mesnard2023}


The source code, which is used for simulations and post-processing scrips, should be easily accessible on GitHub \cite{github} along with containers \cite{Docker_introduction} to reproduce main experiments. The container encapsulates all dependencies, allowing others to replicate the main experiments. Any initial conditions, boundary conditions, and material properties necessary for the simulations should be documented, as well as detailed instructions on how to run the simulations, including necessary command line arguments or configuration settings. Detailed explanations of the theoretical used model are provided in the next chapter is the implementation of code and the interpretation of results are provided in Chapter 3. The implementation done by developing force  third-party libraries or modifying existing ones. After implementation, the model needs to be supported by a verification and validation procedure. Once the implementation is complete the results must be analyzed and interpreted. The grid convergence study also supports the stability of the solver. After that, \textbf{we} tested the solver running multi-spherical bodies interacting with fluid, and the last step was an application to real-world problems to test its effectiveness and usefulness. It was an experiment where a granular mass fell into the water as in works \cite{mao2020resolved} or \cite{shen2022resolved}.
\newpage
\section{Literature review}

We considered a substantial variety of methods and approaches throughout the research process to enhance our understanding of the research methodology and its rationale.

The field of landslide research is vast, with many academic papers and numerous approaches proposed in recent years. The primary criterion for the simulation aspect of this study was to ensure the use of open-source code to support the requirement for reproducibility. As the landslide process involves a solid component interacting with fluid, it was necessary to choose the most suitable method for simulating solid and fluid components with a free surface in an optimized and compatible manner. It is essential to consider the variety of free surface reconstruction methods available, as they differ in terms of numerical stability and the ability to capture the interface between fluids.

\section{Methods for fluid simulation}

Fluid simulation plays a critical role in numerous engineering and scientific fields, ranging from designing more efficient aircraft to predicting weather patterns. Various methods are available for fluid simulation, including Smoothed Particle Hydrodynamics (SPH) \cite{monaghan1994SPH} \cite{gingold1977SPH}, Finite Element Method (FEM) \cite{lewis2004fundamentals}, Volume of Fluid (VOF) \cite{hirt1981volume}, and Lattice Boltzmann Method (LBM) \cite{chen1998lattice}. Each method has its advantages and limitations.

Smoothed Particle Hydrodynamics (SPH) is a Lagrangian method that models fluid as a collection of particles that interact with each other based on their relative positions and velocities. SPH has proven effective in simulating fluid flows with complex geometries and free surfaces \cite{adami2012SPH}, such as ocean waves \cite{barreiro2013SPH} and splashing water \cite{moreira2020SPH}. The method requires a large number of particles to accurately capture the free-surface dynamic, which can result in high computational costs and make it challenging to simulate larger-scale systems. Another area for improvement of SPH is that it relies on artificial viscosity to handle fluid viscosity, which can lead to inaccurate results, especially when the flow is complex \cite{zhang2018dualsphysics}. The artificial viscosity can also cause unwanted damping effects on the fluid motion. Furthermore, SPH struggles with accurately modeling fluid-solid interactions \cite{Dual_SPH2019accuracy}, causing inaccuracy in boundary treatment. The placement of boundary particles, for instance, can affect the accuracy of the results, and it can be challenging to represent complex geometries correctly.

The Finite Element Method (FEM) is another fluid simulation approach. It is an Eulerian method that discretizes the fluid domain into finite elements \cite{FEM}, where a set of equations represents fluid properties. FEM can handle complex geometries and boundary conditions and has been widely used in simulating fluid-structure interactions. However, FEM requires a high level of mesh refinement to accurately capture the fluid behavior near solid boundaries, which can result in significant computational overheads \cite{FEM_2}. Due to this reason, we did not consider this method for our work.

The LBM \cite{begum2008lattice} is a relatively new method that models fluid as a set of particles moving on a lattice. It is a so-called meshless method. LBM has demonstrated excellent performance in simulating fluid flows with complex geometries and boundary conditions and is particularly useful in modeling microfluidic systems \cite{aidun2010lattice}. However, LBM is not the best choice since our application focuses on landslides, and simulating arbitrarily shaped bodies using LBM could be quite complicated.

Finally, VOF is an Eulerian method that tracks the fluid interface by solving a transport equation for each fluid phase with different density parameters. The method conceptually marks cells between $0$ and $1$. Various approaches have been suggested to handle complex geometries, free surfaces, and multiphase flows, such as the Marker and Cell method \cite{mac}, Geometric Reconstruction \cite{VOF_reocnstr}, High-Resolution Interface Capturing \cite{HIRC}, or Level Set \cite{VOF_level_set}. We will discuss these methods more thoroughly in the following sections. VOF, in combination with free surface reconstruction methods, offers several advantages over other methods. These advantages include accurate tracking of the fluid interface, modeling surface tension effects, and maintaining computational efficiency.

After considering the advantages and limitations of each method and the specific requirements of our simulation, we have concluded that the VOF method is the most appropriate choice for our application. But we need to choose an appropriate way for the free-surface reconstruction.

\section{Method for free-surface representation} %free surface simulation

Free-surface simulation took a significant part of research. It is a crucial technique in studying fluid dynamics and is widely used in many fields, such as aerospace, naval architecture, civil engineering, and medical devices. Several methods are available for simulating free-surface flows, each with advantages and limitations. There are two general approaches when using the Finite Volume Formulation (or considering Eulerian formulation): interface-tracking methods and interface-capturing.

The first category includes the Front Tracking Method (FTM)\cite{front-tracking} and Marker and Cell (MAC) Method \cite{mac}. While interface-tracking methods could be effective in some scenarios, they have fundamental disadvantages compared to interface-capturing methods. One of the most important ones is the limited ability to handle topological changes such as merging, breaking, or creating new interfaces, which can be challenging. Such events often require advanced mesh manipulation techniques, which can be difficult to implement and may introduce instabilities. Although CFD is generally complex, interface tracking methods often involve complex algorithms and data structures to track and manage the moving interface mesh or markers explicitly. This can lead to higher computational costs and require more memory and processing time.
Ensuring good mesh quality and resolution near the interface is critical for interface tracking methods. Maintaining an optimal mesh can be difficult, especially in large deformations or high curvature cases. Due to the complexity of managing the interface mesh or markers, interface tracking methods can be more challenging to parallel and scale efficiently on high-performance computing systems.

In contrast, interface-capturing methods like algebraic \cite{algebraicVOF} and geometric VOF \cite{roenby2019isoadvector} formulation, Level Set Method (LSM) \cite{VOF_level_set}. A group of sharp interface methods \cite{sharp-interface} offers a more straightforward and flexible approach to handling complex topologies with potentially lower computational costs. However, they may suffer from a smeared interface representation and require additional techniques to sharpen the interface and maintain mass conservation.

Some well-known geometric reconstruction methods include the Piece-wise Linear Interface Calculation (PLIC) \cite{huang2012piecewise} method, which reconstructs the interface as a line in 2D or a plane in 3D within each interfacial cell. The PLIC is one of the most widely used methods for reconstructing interfaces in VOF simulations. The Height Function Method \cite{height-function} reconstructs the interface by calculating the height of the fluid in each cell, which is then used to derive the interface normal and curvature. This technique is particularly effective in structured Cartesian meshes. Youngs' Method \cite{youngs-method} represents the interface as a piece-wise linear surface within each cell, allowing for time-dependent multi-material flow simulations with significant fluid distortions. The isoAdvector \cite{roenby2019isoadvector} method involves two main steps: exploiting an iso-surface concept for modeling the interface inside cells in a geometric surface reconstruction step and modeling the motion of the face-interface intersection line for a general polygonal face to obtain the time evolution within a time step of the submerged face area. As a plus, it has an open-source implementation as a part of OpenFOAM \cite{jasak2007openfoam}.

These are just a few examples of the geometric reconstruction schemes used in multiphase flow simulations. Each method has its advantages and disadvantages, and choosing the most suitable method depends on factors like the problem's complexity, computational resources, and accuracy requirements.

Algebraic reconstruction schemes are used in interface capturing methods, particularly within the context of the Volume of Fluid (VOF) method. These methods compute the fluxes algebraically without the need for geometric reconstruction of the interface. Some well-known algebraic reconstruction methods include Compressive Schemes, which use the information of the interface's orientation (interface normal) with respect to the cell face to compute the face flux. One example of a compressive scheme is the High-Resolution Interface Capturing (HRIC) method \cite{HIRC}. Tangent of Hyperbola for Interface Capturing (THINC) \cite{THINC} schemes utilize a hyperbolic tangent function to represent the phase indicator function. The volume fraction is then obtained using a numerical approximation, such as volume-averaged, polynomial, or hyperbolic tangent representation. The CICSAM (Compressive Interface Capturing Scheme for Arbitrary Meshes)\cite{CICSAM} is an extension of the compressive VOF method for unstructured meshes. It uses the orientation of the interface normal to compute the face flux. Slope-Limiter-Based Methods \cite{liu2021new} use slope limiters to control the steepness of the volume fraction gradient within the cells, which helps to avoid numerical oscillations and maintain the sharpness of the interface. Multidimensional Universal Limiter for Explicit Solution (MULES)\cite{MULES} is a numerical scheme where the advection term is modified to compress the surface and reduce smearing. It helps achieve a higher-order scheme for a more accurate advection at the surface. These algebraic reconstruction schemes offer varying degrees of accuracy and computational efficiency. The choice of the most suitable method depends on factors like problem complexity, computational resources, and accuracy requirements. Although based on experiene of use this approach in OpenFoam \cite{roenby2019isoadvector} the method does not fully capture the movement of the free surface and the formation of droplets.

The Level Set method is a popular numerical technique used for tracking interfaces and shapes in various applications, including two-phase flow simulations. It applies widely to multiple problems, including fluid mechanics, computer graphics, and image processing. The method has its advantages, such as accurate computation of interface properties. The technique allows precise calculation of normals and curvature, which are crucial in many interfacial flow applications. Another benefit of the method is implicit interface representation, which makes it easier to handle complex topologies such as merging, breaking, and self-intersecting interfaces. The Level Set method can be easily extended to work with Adaptive Mesh Refinement (AMR), allowing for better resolution in areas of interest and reduced computational cost.

On the other hand, several disadvantages may make the method not the first choice. One of the most significant issues is that it does not inherently conserve mass for each phase, a critical requirement in the numerical modeling of realistic two-phase flows. Although there are techniques to limit mass loss, the problem cannot be eliminated entirely. Moreover, solving the Level Set equation can be computationally expensive, especially for large-scale problems or those requiring high accuracy, and the Level Set function needs to be reinitialized frequently to maintain a signed distance function, which can introduce additional computational overhead and may lead to a loss of accuracy. Coupling the Level Set method with other physics, such as fluid flow, can be challenging and may require additional numerical techniques to address issues like mass conservation or surface tension.

Based on the literature review, the most general and accurate method is the geometric VOF, which has implementation as a OpenFoam solver \cite{roenby2019isoadvector} which called IsoAdvector, was chosen for free surface reconstruction. Because of the complexity of the technique, it is necessary to describe it in more detail, which will be done in the next chapter.

The isoAdvector method is unique for a couple of reasons:: First, it uses a special way to draw the boundary line inside a cell when we're looking at how a liquid's surface changes. Secondly, it tracks how this outlined surface moves across the cell's boundary over time. By figuring out how much of the boundary gets wet during a certain period, we can precisely calculate the amount of liquid that's moved through it.

%challenges include, but are not limited to (1) enforcing mass, momentum and kinetic energy conservation, (2) modeling discontinuities in properties across the interface, especially large jumps in density, (3) handling complex topologies and separation of scales, (4) achieving robustness for simulation of realistic flows, (5) accurately implementing surface tension forces.


\section{Methods for granular media simulation}

Several methods are available for simulating granular media, each with advantages and limitations. These methods include the Discrete Element Method (DEM)\cite{cundall1979discrete}, Finite Element Method (FEM), Smoothed Particle Hydrodynamics (SPH)\cite{monaghan1994SPH}, Lattice Boltzmann Method (LBM) \cite{chen1998lattice} \cite{begum2008lattice}, and Molecular Dynamics (MD).

DEM is a method widely used for studying the behavior of granular materials under different loading conditions. In this approach, granular media are modeled as a collection of discrete particles interacting with each other through contact forces. One of DEM's strengths is its ability to capture the details of individual particle interactions and contact forces, making it suitable for studying granular materials with complex geometries and boundary conditions.

In contrast, FEM is useful for simulating large-scale systems with solid bodies, complex geometries, and boundary conditions. While FEM can capture the behavior of granular materials under different loading conditions, it is less accurate in capturing the details of individual particle interactions.

The Smoothed Particle Hydrodynamics Method could be used not only for fluid simulation but also for predicting the behavior of granular media as a fluid with individual particles representing the grains. This method is well-suited for simulating the dynamics of granular flows and mixing processes. However, its accuracy in capturing the detailed contact forces between individual particles is limited, which restricts its applicability to certain types of granular materials.

Lattice Boltzmann Method falls into the same category as the SPH group of methods. It is an efficient method for simulating large-scale systems with complex geometries and can accurately capture the fluid-like behavior of granular materials. However, it can be computationally expensive for high-resolution simulations, and its accuracy in capturing the detailed contact forces between individual particles is limited.

Molecular Dynamic Methods help study the behavior of granular materials at a microscopic scale and can capture the effects of thermal fluctuations and chemical reactions. However, it is computationally expensive and may need to be more practical for simulating large-scale granular systems.

Considering each method's strengths and limitations, we have chosen DEM featuring multi-spherical bodies, called clumps in the literature, as our method for simulating granular media. DEM can capture the details of individual particle interactions and contact forces, and adding clump features allows for the modeling of cohesive forces between particles, which can be crucial in certain types of granular materials. 

\section{Coupling approach.}

Coupling methods are commonly used in simulations where multiple physical phenomena are involved. There are different ways to couple simulations; each method has advantages and disadvantages. Some common coupling methods include one-way coupling, two-way coupling, and fully coupled simulations.

In one-way coupling, one simulation provides input to another, but the second simulation does not affect the first. This method is useful when one simulation has a negligible effect on the other simulation.

In two-way coupling, which some time called strong coupling, both simulations interact with each other, and the results of each simulation affect the other. However, the interactions between the simulations are limited to specific points in time, and the simulations do not interact with each other at every time step.

The two simulations are fully integrated in a fully coupled way, and the results of both simulations are obtained simultaneously. The simulations interact at every time step, and each simulation considers the other simulation's effect.

In our work, we have chosen to use two-way coupling. In this way, the method would allow us to simulate the interactions between different physical phenomena while being computationally efficient. This choice will provide the necessary level of accuracy while keeping the computational costs reasonable.

To run this type of simulation code based on CFDEMcoupling \cite{kloss2011liggghts} was chosen as well, and their two-way coupling approach was chosen as a base. There are a couple of main reasons why \textbf{we} decided to use this approach:
\begin{itemize}
    \item CFDEM is open source project with good support
    \item Project based on OpenFoam - open source C++ programming library. Freedom to choose the type of discretization, boundary conditions, methods to project solid body on Eulerian mesh, and even what approach to use to model free surface. 
    \item DEM part is open source, based on molecular dynamic package \cite{LAMMPS} with options of creating clump as a body and possibly implementing force interaction models for granular media.
\end{itemize}

Verification and validation procedures of CFDEMcoupling \cite{kloss2011liggghts} already done by \cite{hager2014cfd} and \cite{balachandran2021resolved} for one phase fluid interacting with solids. The next step is to implement multiphase and apply the method to the real-world problem.

\section{Reproducibility}

Despite the fact that in recent years, there have been a couple of academic papers published using similar approach \cite{nan2023high}, \cite{mao2020resolved}, in current work used more efficient method for free surface simulation and probably different way of implementing force which communicate back and forth between solid and fluid parts. The authors of published works do not share their source code; data from simulations and experiments used for verification are impossible to repeat because of a lack of experiment parameters. That makes works unreproducible and could deprive readers of understanding how the proposed approach works. One goal of the current research was to ensure the application is reproducible. A container for the solver, scripts, data, and configuration files are available on GitHub and dockerhub (\textbf{links}) and Jupyter notebooks in the manuscript repository.\textbf{gonna publish plots later, but maybe need to do it rn!!!!}

\section{Thesis outline}

The research project aims to simulate the behavior of fluids, granular media, and their interactions. To achieve this goal, we must carefully choose the appropriate numerical methods for each simulation aspect.

We considered various approaches for fluid simulation, such as SPH, FEM, VOF, and Lattice Boltzmann. After analyzing the pros and cons of each method, we concluded that the geometric VOF method with the isoAdvector approach for phase reconstruction is the best choice.

Next, we evaluated several techniques for free surface simulation, including level-set, front-tracking, and a customized version of the LSM. After careful consideration, we decided to use the isoAdvector approach with some modifications to meet the specific needs of the simulation.

Finally, we looked at various methods for granular media simulation, such as DEM, SPH, and lattice-based models. Based on the analysis of each method's strengths and weaknesses, DEM with multi-spherical body features looks the most appropriate. With two-way coupling method for the simulation interaction of the solid part and the two-phase fluid. This method allows the most accurate modeling of the interactions between the media.

We implemented and run simulations with the method developed and described in the previous steps to validate and verify simulation results. This will include parallel options for simulation and will be supported with verification and validation procedures.
Ultimately, the results of the simulations applied to a real-world problem for demonstration the practical applications of this research.

In conclusion, this research project represents an analysis of the two-phase fluid, free-surface, and granular media simulation methods for simulation model. This study's findings can advance our understanding of the interaction of fluids and granular media, with real-world applications in fields such as engineering and geology, and provide reproducible results that could be improved and used by other researchers.