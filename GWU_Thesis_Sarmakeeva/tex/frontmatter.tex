% !TEX root = ../thesis-sample.tex

% --------- FRONT MATTER PAGES ---------------------

% Title of the thesis
\title{A Resolved CFD-DEM Coupling Method for Simulation Two-Phase Fluids Interaction with Arbitrary Shaped Bodies.}
% capitalize significant words!

% Author name
\author{Anastasiia Sarmakeeva}

% Previous degrees
\bachelordegree{Specialist's degree}
\bsdepartment{Applied Mathematics and Computer Science}
\bsschool{Udmurt State University}
\bsgrad{July 2014} % "month year"

\masterdegree{M.S.}
\msdepartment{Your graduate old department}
\msschool{Your graduate alma mater}
\msgrad{Month YYYY}  % "month year"
% you can show or hide the MS degree line
%\showmsdegree
\hidemsdegree

% PhD degree commands
% Committee
\showcommitteepage % hide this page if you're doing a MS thesis
%\hidecommitteepage 


% define COMMITTEE information

% in general, note that administrative titles are not used, instead use "professorial titles"?

% Chair must be entered separately for formatting reasons.
\chair{Lorena A. Barba}
\department{Mechanical and Aerospace Engineering}
\chairtitle{Professor of Mechanical and Aerospace Engineering}

% uncomment and use the below if you have a co-chair. note the code reacts to the \cochair{} , not the \cochairtitle{}
% \cochair{Co-Chair Person}
% \cochairtitle{Just-as-Amazing Professor of \insertdepartment}

\phdschool{The School of Engineering and Applied Science}

\committee{ 
% director first
Lorena A. Barba, Professor of Mechanical and Aerospace Engineering, Dissertation Director

% remember to add a space between committee members
\vspace{\baselineskip}
Elias Balaras, Professor of Engineering and Applied Science, Committee Member \hfill
    

\vspace{\baselineskip}
Kausik Sarkar, Professor of Mechanical and Aerospace Engineering, Committee Member \hfill

\vspace{\baselineskip}
% note you shouldn't write "The George Washington University" at all- it is implied
Thomas Lichtenegger, Department of Particulate Flow Modelling, Committee Member \hfill

\vspace{\baselineskip}

Full Name, Title, Dissertation Director/Dissertation Co-Director/Committee Member \hfill

\vspace{\baselineskip}

% external examiner
Name of External Examiner, Professorial Title, Name of External University (or Name, Job Title, Name of External Company), Committee Member  % include university or company of any external examiner! but still "committee member"
}

\phdgrad{CHANGE January 8, 2024}  % Month DD, YYYY
\defensedate{December 18, 2023}  % Month DD, YYYY
% Year of completion for copyright page and perhaps other places
\year=2023

% Copyright page
%\copyrightholder{Someone else}

% Dedication
\dedication{ %
\emph{To everyone who supported me on this journey.}
}

% Acknowledgments
\acknowledgments{
    For me, this journey has been longer and more complex than anyone could have ever imagined. First and foremost, I'd like to express my deepest gratitude to my family — my mom, dad, and brother. I couldn't have reached this point without your unwavering support.
    
I must also extend my heartfelt thanks to my advisor, Professor Lorena Barba. Her introduction to Almadena Chtchelkanova solidified my decision to complete my work in the U.S. under her guidance. I am immensely grateful for her support throughout my time at George Washington University.

I also owe a debt of gratitude to Professor Thomas Lichtenegger. His support at the PFM lab significantly enhanced the computational aspect of my research, and I gained invaluable insights from our meetings.

Special acknowledgment goes to Zephra Coles, the Department Operations Lead. Her outstanding management skills have made my paperwork at GWU far more manageable.

I want to extend my thanks to my lab mates: Paulina Rodriguez, Ting-Gyu Wang, Olivier Mesnard, Natalia Clementi, and Pi-Yueh Chang.
Further thanks go to researchers Leonid Tonkov, Alena Chernova, and Alexander Novikov. Their expertise and support were invaluable as I embarked on my research journey.

Last but not least, I want to thank my partner, Lawrence Kirk, your emotional and intellectual support has been my rock. My Russian-speaking graduate friends, Anna Medvedeva, Maria Sidulova, and Anna Webber. Your support has made my journey in a foreign country significantly smoother.
}

% -----------------------------------------------------------------
% Typically only one of Preface/Foreword/Prologue would be in your thesis.
% To choose one simply delete the others and they will automatically disappear

% Preface
\preface{
    This is the preface. 
    It's another front matter page that offers additional detail into your work.
    Typically, only one (preface OR prologue OR foreword) is used. 
    You can remove the other sections by deleting them inside \texttt{tex/frontmatter.tex} or using the appropriate show or hide commands.
}

\prologue{
    This is the prologue. 
    It's another front matter page that offers additional detail into your work.
    Typically, only one (preface OR prologue OR foreword) is used. 
    You can remove the other sections by deleting them inside \texttt{tex/frontmatter.tex} or using the appropriate show or hide commands.
}

\foreword[2]{
    This is the foreword. 
    It's another front matter page that offers additional detail into your work.
    Typically, only one (preface OR prologue OR foreword) is used. 
    You can remove the other sections by deleting them inside \texttt{tex/frontmatter.tex} or using the appropriate show or hide commands.
}
% ----------------------------------------------------------------------

% commands to show or hide front matter pages

\showcopyright
\showabstract
\showcommitteepage
\showdedication
\showacknowledgments
\showpreface
\hideprologue
\hideforeword

% Commands to hide or show lists of figures, tables, etc.
\showtableofcontents
\showlistoffigures
\showlistoftables
\hidenomenclature

% ^ NOTE! April 4 hack for April 15 deadline - if any of these are missing from your TOC and there should be entries there, see lines 1435-1447 of the class file for an example. You need to wrap the appropriate function call adding a call to addcontentsline




% --------- ACRONYMS and SYMBOLS ------------------------------
% TODO Deprecate the entire acronym package and switch to glossaries

% You can either use the acronym or glossaries package (both work)
% Definition of any abbreviations used.
\abbreviations{
    \acro{VOF}{Volume of Fluid}
    \acro{FVM}{Fine Volume Method}
    \acro{MPI}{Message Passing Interface}
    \acro{FEM}{Finite Element Method}
    \acro{DEM}{Discrete Element Method}
    \acro{FSI}{Fluid Structure Interaction}
    \acro{SPH}{Smooth Particle Hydrodynamic}
    \acro{LBM}{Lattice Boltzman Method}
    \acro{MD}{Molecular Dynamic}
    
}
% call an abbreviation using \ac{abbrev}


% if you want acronym (simpler) then change these to show
%\showlistofabbreviations
%\showlistofsymbols

% if you want glossaries (more powerful) then leave above as hide
% GLOSSARIES package options - automatically turns off front pages from acronym package

% acronymns and symbols are basically the same, but there are two provided 
% locations where they can show up
\setabbreviationstyle[acronym]{long-short}
\setabbreviationstyle[abbreviation]{long-short}
%\makeglossaries
% you can hide/show the glossaries page
%\showglossarieslistofabbreviations
%\showglossarieslistofsymbols
%\showglossariesglossaryofterms

% acronyms defined in glossaries
\newabbreviation{crtbp}{CRTBP}{Circular Restricted Three Body Problem}
\newabbreviation{lidar}{LIDAR}{Light Detection and Ranging}
\newabbreviation{hideme}{HIDEME}{Hide One of These Abbreviation Systems; this is from Glossaries}

% defining abbreviations like this allows for autocompletion
\newglossaryentry{filo}{
    name={FILO},
    type=\glsxtrabbrvtype,
    description={first in last out},
    first={first in last out (FILO)}
}

% glossary entries
\newglossaryentry{linux}{
    name=Linux,
    description={is a generic term referring to the family of Unix-like computer operating systems that use the Linux kernel},
    plural=Linuces
}

\newglossaryentry{matrix}{
    name={matrix},
    plural={matrices},
    description={rectangular array of quanttities}
}

% symbols
\newglossaryentry{M}{
    type=symbols,
    name={\ensuremath{M}},
    sort=M,
    description={a \gls{matrix}}
}

\newglossaryentry{F}{
    type=symbols,
    name={\ensuremath{F}},
    sort=F,
    description={External Force}
}

% Some abstract text
\abstract{
Landslides, causing over 5000 deaths annually, not only represent a tragic loss of life but also bring devastating economic consequences. With increasing flood frequency and severity exacerbated by drought and global warming, understanding and predicting landslides involving granular media, water, and the air is crucial. Our research aims to study the interaction of particles, rocks, and boulders with water.

For simulations, we use CFD methods combined with the Discrete Element Method (DEM) to represent solid bodies of arbitrary shapes. To improve the simulation model, we developed a force model based on the open-source code CFDEMcoupling\cite{kloss2012models}. We use a resolved CFD-DEM approach, where the solid body takes more than a computational cell. This simulation helps to track interactions between fluids and solids, such as the movement of soil or rocks falling into water.

The fluid and solid components are discretized using the Eulerian and Lagrangian frameworks. The solid component is coupled using the Immersed Boundary Method, where a solid body projects into the CFD mesh. For the free surface simulation in the current work, we use the sharp interface method based on isoAdvector solver \cite{roenby2019isoadvector} and the DEM for solid body simulation. Validation and verification showed that the solver performs according to our expectations. We conducted more detailed simulations based on those initial results, including a bouncing body test, demonstrating the method's stability and reliability. Then, we ran the simulation with multispherical body interactions and two-phase fluids, showing that the method could effectively handle complex interactions between solid and liquid phases and the dynamics of multiple solid bodies. It can track the motion and interaction of bodies with non-uniform shapes through the fluid, which involves complex boundary conditions and potentially non-linear material behavior. Results also show that the solver is effective for parallel computations by the decreased computation times with increased processor counts. We provide the solver's documentation online, and a container application ensures reproducibility. 

This advanced computational capability, demonstrated by our solver, is crucial for accurately modeling and understanding real-world phenomena such as landslides, sediment transport, and various particulate flow problems, thereby contributing significantly to computational fluid dynamics and disaster risk management.

}
