\chapter{Conclusion} \label{chap:conclusion}
Landslides kill thousands of people every year. Their frequency and severity are escalating, and progressive global warming is making them an ongoing concern. This research provides a numerical tool that could help predict landslide processes involving complex interaction with heterogeneous granular particles, air, and water, which is usually the case in coastal areas. 

This work examines existing algorithms for presented problem reviewing and considering many different approaches for each part of the simulation process, which is provided in Chapter \ref{chap:intro}. Because landslides in coastal areas are a complicated process, the problem was divided into several parts, with a deliberate choice of method for solid part, fluid, and multiphase flow simulation.

First, it was decided that the most efficient method for solving this type of problem could be the Fluid-Structure Interaction (FSI) method, which raised the question of the method of coupling. The closest to real-world issues is two-way coupling, where two different methods for solid and fluid parts exchange data each time step.

Then, an approach for two-way coupling was considered. The immersed boundary method used to project solid bodies to Eulerian mesh. Fluid simulation was considered to be a mesh-based or mesh-less simulation method. Although mesh-less methods are rising in popularity due to mass conservancy, when solid particles are included, they do not save the same advantages, making the approach more expensive and suffering from issues with numerical stability and accuracy, particularly near boundaries. This limitation led to the choice of mesh-biased Volume of the Fluid method, which is widely used in multiphase flow simulation areas.

Mesh-based methods for multiphase flow simulation have a long history of development and use, making them well-understood. They have extensive literature and robust software support. The Finite Volume Method is known for its accuracy in representing complex geometries, especially near boundaries. 

To simulate multiphase flow, we used the Volume of Fluid method. This choice raised a question about a technique for resolving free surfaces. Considering the stability of the numerical scheme in the chosen open-source programming library, OpenFOAM, where the MULES scheme was initially available, isoAdvector was incorporated into the developed solver due to the known disadvantages of the MULES scheme for better numerical stability.

In contrast to standard unresolved coupling methods, the resolved coupling approach offers a mesoscale view of fluid flow in porous media without relying on empirical drag-force models. Through a series of validation tests and case studies, we've examined various scenarios, including single and two-phase fluids and interactions with both spherical and non-spherical particles.

For verification and validation purposes, we run grid convergence analysis with different mesh resolutions and for various time steps. Solutions were compared with already existing results from numerical studies \cite{nan2023high}. Then, to test the free surface reconstruction algorithm, we tested a falling sphere into water with air-fluid-solid interaction and compared results with \cite{pathak20163d}; results showed good agreement. As a last step, we ran the experiment of falling into fluid, multiple spherical bodies, which showed the stability of the scheme during fluid-solid interaction.

\textbf{Key Findings}:
\begin{itemize}
    \item Our enhanced resolved coupling solver is validated for two-phase fluids. Tests involving a single settling sphere indicate that the dynamic fluid force on the particle aligns well with simulation data. The model also effectively captures the influence of the air-water interface on particle settling.
    \item For particles of complex shapes, a multi-sphere model is employed. Tests, including clumped particle settling, confirm the solver's ability to model fluid interactions with clumped particles.
   \item The solver demonstrated efficient parallel performance, achieving a speedup of more than 80\%, which meets the commonly accepted threshold for parallel efficiency.
    \item The solver is also capable of modeling non-spherical particles, as evidenced by particle settling case.
    \item Mesh decomposition affects the process of simulation for the single sphere. However, it might indicate the presence of a bug in the code. As well as the coupling approach is affected by the exchange time between solvers for DEM and CFD parts.
    \item The model successfully captures phenomena, such as air-water interaction and the multiple clumps with the density of stone material. This shows that the developed tool for simulation could be extended for the simulation of the large-scale process of landslide.
\end{itemize}

\textbf{Future Directions}:

Despite its successes, the model has limitations. Computational efficiency is a concern due to the need for small computational cells. While parallel computing helps to speed up computational time, it could affect the simulation results, as was shown in a falling sphere example. Dynamic mesh refinement could further improve efficiency and reduce computational time. The method for projecting solid bodies to a CFD mesh could also be improved to better define solid bodies. Also, for computing solid body interaction, additional interaction models could be developed since in the current study, only the Hertz contact model was applied. Other free surface reconstruction methods could be used as well for better numerical stability.

Additionally, the current study is limited to Reynolds numbers up to several hundred. Future work could explore the model's applicability to high Reynolds number scenarios. The model could also be extended to realistic coastal engineering problems like embankment erosion and protection with different contact forces for solid body interactions.

Nevertheless, despite the problem's complexity and the numerous challenges encountered, such as the unavailability of ready-made solutions and making our research accessible to others. Our results are promising due to using a more efficient than in-house OpenFOAM approach for free surface reconstruction. The future developments hold significant potential.